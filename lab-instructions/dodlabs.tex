\documentclass[fleqn, article, a4paper]{memoir}

% Swedish.
% \usepackage[T1]{fontenc}
% \usepackage[utf8]{inputenc}
% \usepackage[swedish]{babel}
\usepackage{pdfpages}

\usepackage[shortlabels]{enumitem}
\usepackage{selthcsexercise}
\usepackage{microtype}

% Utilities.
\usepackage{graphicx}
\usepackage[hyphens]{url}
\usepackage[swedish]{varioref}
\usepackage{listings}
\usepackage{fontspec} % To enable use of ttf fonts
\usepackage{enumitem} % For customizing lists

\usepackage{amsmath}  % For \text{}
\usepackage{pifont}   % Provides the star symbols
\usepackage[swedish]{babel}
\usepackage[autostyle, swedish=quotes]{csquotes}

%---------------------------------------------------------------

% Gemensam hjälpmakro som bara ritar stjärnorna
\newcommand{\stars}[1]{%
  \ifnum#1>0 \ding{72}\else \ding{73}\fi
  \ifnum#1>1 \ding{72}\else \ding{73}\fi
  \ifnum#1>2 \ding{72}\else \ding{73}\fi
  \ifnum#1>3 \ding{72}\else \ding{73}\fi
  \ifnum#1>4 \ding{72}\else \ding{73}\fi
}

% Blockvariant (som tidigare)
\newcommand{\difficulty}[2][]{%
  \emph{Svårighet:} \stars{#2}%
  \ifx\relax#1\relax \else \hspace{3mm}(\textit{#1})\fi
  \par\noindent
}

% Inline-variant för löpande text
\newcommand{\diffinline}[2][]{%
  \stars{#2}%
  \ifx\relax#1\relax \else \,(\textit{#1})\fi
}



% Set section count to -1, so the first section (which is hidden) becomes 0,
% and thus the first section actually becomes 1
\setcounter{section}{-1}

%---------------------------------------------------------------
\newenvironment{Hemarbete}%
{\begin{Assignments}[H]}{\end{Assignments}}

\newenvironment{Kontrollfragor}%
{\begin{Assignments}[F]\tightlist}{\end{Assignments}}

\newenvironment{Datorarbete}%
{\begin{Assignments}[D]}{\end{Assignments}}

\newenvironment{DatorarbeteCont}%
{\begin{Assignments}[D]\setcounter{Ucount}{\theSavecount}}{\end{Assignments}}

\newenvironment{Extrauppgifter}%
{\begin{Assignments}[E]}{\end{Assignments}}

\newenvironment{Deluppgifter}%
{\begin{enumerate}[a)]\firmlist}{\end{enumerate}}

\newcommand{\Förberedelser}{
\subsection*{Förberedelsefrågor}
Dessa frågor hjälper dig att repetera viktiga begrepp 
och bedöma om du har förberett dig tillräckligt inför laborationen. 
Använd dem som stöd för dig själv---du behöver inte lämna in svaren.
}

% \newcommand{\file}[1]{{\slshape\fontfamily{ppl}\selectfont #1}}
\newcommand{\file}[1]{\texttt{#1}}

\newcommand{\commandchar}[1]{\textsc{#1}}

\newcommand{\progname}{c3pu}
\newcommand{\progversion}{v1.0}
\newcommand{\progfilename}{\progname-\progversion.jar}

% Define the hint command
\newcommand{\hint}[2][]{\par\halfblankline\noindent\textbf{Ledtråd\ifx#1\empty\else\space#1\fi}: #2\par}

\lstset{
	escapeinside={(*@}{@*)},
}

\setlist[description]{
	style=multiline,
	align=right,
	itemsep=1mm,
	parsep=0mm,
	rightmargin=1cm,
}

% Section styles.
\setsecheadstyle{\huge\sffamily\bfseries\raggedright} 
\setsubsecheadstyle{\Large\sffamily\bfseries\raggedright} 
\setsubsubsecheadstyle{\normalsize\sffamily\bfseries\raggedright} 

\setsecnumformat{} % numrera inte laborationerna
\renewcommand{\thesection}{\arabic{section}} % för referenser till laborationerna
\renewcommand{\thefigure}{\arabic{figure}}

%*****************************************************************
\author{}
\begin{document}

\clearpage
\thispagestyle{empty} % Removes page number
\vspace*{30mm}
\begin{center}
	\sffamily
	\renewcommand{\baselinestretch}{1.1}
	\Huge\bfseries Datorlaborationer \\[5mm]
	EDAB05 \\[2mm]
	\LARGE\bfseries	Datorer och datoranvändning \\[7mm]
	\large Lunds universitet, LTH --- \the\year
\end{center}
\clearpage


%*****************************************************************
\courseinfo{Datorer och datoranvändning}{\the\year}
\maketitle
\thispagestyle{titlepage}
\vspace{-4cm}
\section*{Datorlaborationer, datorer och datoranvändning}

Datorlaborationerna ger exempel på tillämpningar av det material som behandlas under dod-delen av kursen. För laborationerna gäller följande:

\begin{itemize}
    \item Laborationerna är obligatoriska. Det betyder att du måste bli godkänd på alla uppgifter under ordinarie laborationstid. Om du är sjuk vid något laborationstillfälle ska du anmäla detta till kursansvarig (\url{mattias.nordahl@cs.lth.se}) före laborationen. Gör sedan uppgiften på egen hand och redovisa den på en resurstid eller vid ditt nästa laborationstillfälle.

    \item Uppgifterna i laborationerna ska lösas individuellt. Regler för samarbete finns på nästa sida.

    \item Varje laboration består av två delar: förberedelser (hemarbete) och datorarbete. Innan du kommer till laborationen ska du ha gjort de uppgifter som är markerade som förberedelser, och gärna också ha skummat igenom vad som står under datorarbete. Du ska dessutom ha gått igenom förberedelsefrågorna i början av varje laboration.
    
    \item Förberedelsefrågorna kommer inte att kontrolleras av handledaren, utan är till för att du själv ska kunna bedöma om du har förberett dig tillräckligt.

	\item Svaren till förberedelsefrågorna finns i laborationshandledningen eller i annat material som hänvisas till. Du kommer också att behöva reflektera själv utifrån materialet.

    \item Under laborationerna får du ta hjälp av allt kursmaterial och andra resurser för att lösa uppgifterna. Ta också hjälp av handledaren om du fastnar eller behöver stöd.

    \item Observera att laborationerna inte bara syftar till att testa dig, utan är ett \textbf{inlärningsmoment}! Diskutera gärna och ställ frågor till handledaren om allt du tycker är knepigt eller vill veta mer om.

    \item Om du hittar något i uppgifterna eller andra anvisningar som är felaktigt eller oklart uppskattar vi om du meddelar detta till \url{mattias.nordahl@cs.lth.se}.
\end{itemize}



\newpage
\subsection*{Riktlinjer för inlämningsuppgifter och laborationsuppgifter}
Bland LTHs gemensamma regler finns följande föreskrifter:

\begin{itemize}\tightlist
	\item Inlämningsuppgifter skall fullgöras individuellt om det inte särskilt anges att de skall fullgöras i grupp.
	\item Vid arbete i grupp bestämmer ansvarig lärare om gruppindelningen och ändringar av denna. Arbetet skall utföras av dem som ingår i gruppen.
	\item Det är tillåtet att diskutera uppgifterna och tolkningen av dessa med utomstående på ett allmänt plan men inte att få hjälp med de konkreta lösningarna.
	\item Det är inte tillåtet att kopiera annans eller annan grupps lösningar helt eller delvis. Det är inte heller tillåtet att kopiera från exempelvis litteratur eller Internet. Vid citat skall källan tydligt anges.
	\item Väsentlig hjälp, av annan än lärare på kursen, för att genomföra en uppgift skall redovisas i redogörelsen eller på annat tydligt sätt. Detsamma gäller om man använt någon annan form av hjälpmedel som läraren inte kan förutsättas känna till.
	\item Institutionerna kan komplettera dessa regler skriftligen i samband med kursstarten, exempelvis i ett kursprogram.
\end{itemize}

\n Vid institutionen för datavetenskap gäller även följande kompletteringar/förtydliganden av reglerna:

\begin{itemize}\tightlist
	\item Reglerna om arbete i grupp ovan tillämpas för alla obligatoriska moment som utförs i grupp, det vill säga även laborationer och projektarbeten.
	\item Då arbete görs i grupp skall alla gruppdeltagare delta i arbetet.
	\item Hjälp från annan med handhavandet av apparatur, utnyttjandet av datorsystem och givna datorprogram behöver inte redovisas.
\end{itemize}

\n Kontakta ansvarig lärare om du är osäker på om viss hjälp är tillåten eller inte!

\blankline
\n Institutionen tillämpar dessa riktlinjer på alla kurser. Om vi är övertygade om att fusk skett så överlämnar vi ärendet till universitetets disciplinnämnd för vidare åtgärd. Finner disciplinnämnden de studerande skyldiga är påföljden upp till sex månaders avstängning från universitetet och högskolan.

\newpage
\section{Tips}
\label{sec:tips}

Här samlas kort information och hjälp som kan vara användbar under laborationerna. Vid problem och frågor under laborationerna, kolla gärna först om din fråga finns besvarad här.

\subsection{Lab 1 --- Linux/Unix}
\begin{itemize}
    \item \texttt{Vad menas med att \enquote{gå till} en katalog?}\\
          Att \enquote{gå till} en sökväg innebär att ändra ditt \texttt{working directory}. (ILL~1.7)

    \item \texttt{Varför börjar vissa sökvägar med \enquote*{/}?}\\
          Detta indikerar att sökvägen är \emph{absolut} och börjar från rotkatalogen. Ofta är det smidigare att använda \emph{relativa} sökvägar, vilket innebär att de börjar från den katalog du för tillfället befinner dig i (ditt \emph{working directory}).

    \item \texttt{Om jag använder min egen laptop, hur kommer jag åt filer på skoldatorerna?}\\
          Du kan använda \code{ssh} för att logga in och arbeta på skoldatorerna (ILL~3.7), eller kopiera filer från skoldatorerna till din egen dator med \code{scp} eller \code{sftp} (ILL~4.2).
\end{itemize}

\subsection{Lab 2 --- Versionshantering med Git}
\begin{itemize}
    \item \texttt{Hur stänger jag vim!?}\\
          Om du råkat öppna texteditorn \code{vim} (t.ex. genom att göra en git commit utan att ha ställt in en annan editor) kan du stänga den genom att skriva \code{:q} (kolon följt av \code{q}) och sedan trycka \code{Retur}. Om du redan provat annat kan du ha kommit in i något av Vims olika lägen. Tryck då först \code{Esc} för att återgå till det \enquote*{normala läget}.

    \item \texttt{Hur arbetar jag med texteditorn nano?} \\
          \code{nano} är en terminalbaserad texteditor. Den körs alltså direkt i terminalen utan att öppna nya fönster, vilket kan vara fördelaktigt ibland. I editorn kan du flytta markören med piltangenterna och skriva text som förväntat. Längst ned i terminalen visas också vanliga operationer och motsvarande knappkombinationer. Där används tecknet \texttt{\^{}} (den lilla \enquote*{hatten} vid sidan om \commandchar{return}) för att betyda Ctrl-knappen. Använd t.ex. \commandchar{control-o} för att spara (Write Out), \commandchar{control-x} för att avsluta eller \commandchar{control-g} för mer hjälp.

    \item \texttt{Hur ändrar jag min standardeditor för git?}\\
          Du kan ändra din standardeditor för git med kommandot:
          \begin{Code}
              git config --global core.editor nano
          \end{Code}
          Byt gärna ut \code{nano} mot en editor du föredrar och som är installerad på systemet, t.ex. \code{"code~--wait"} (för Visual Studio Code). Notera att vissa editors, t.ex. \code{code}, kräver extra flaggor för att fungera korrekt tillsammans med git.

    \item \texttt{Hur konfigurerar jag separata SSH-nycklar för olika tjänster?}\\
          Ur säkerhetssynpunkt är det bra att använda olika SSH-nycklar för olika tjänster, t.ex. en nyckel för GitHub och en annan för GitLab. Du kan skapa flera nyckelpar med \code{ssh-keygen} (acceptera standardnamn och döp sedan om filerna) och sedan konfigurera SSH-klienten att använda rätt nyckel för varje tjänst genom att redigera filen \file{\textasciitilde/.ssh/config}. Om \file{config}-filen inte finns kan du själv skapa den.

          Exempel på konfiguration med olika nycklar för GitHub, GitLab och LTHs studentdatorer:
          \begin{lstlisting}{}
Host github.com
    User git
    HostName github.com
    IdentityFile ~/.ssh/id_ed25519_github

Host gitlab.com
    User git
    HostName gitlab.com
    IdentityFile ~/.ssh/id_ed25519_gitlab

Host student.login.lth.se
    User xy1234zz-s
    HostName student.login.lth.se
    IdentityFile ~/.ssh/id_ed25519_student
          \end{lstlisting}

          Här anger \code{IdentityFile} sökvägen till den privata nyckeln som ska användas för respektive tjänst. Notera att filen \file{config} inte har någon filändelse och att den måste ha rätt behörigheter --- läs- och skrivbehörighet endast för användaren. Ändra t.ex. med:
		  \begin{lstlisting}{language=bash}
chmod 600 ~/.ssh/config
		  \end{lstlisting}

          \textbf{Notera} också att du behöver logga in på respektive tjänst och lägga till den publika nyckeln i deras system.
\end{itemize}

\subsection{Lab 3 --- \LaTeX}
\begin{itemize}
    \item \texttt{Hur skriver jag specialtecken i \LaTeX{}?}\\
          Vissa tecken har särskild betydelse i \LaTeX{}. För att faktiskt skriva dessa tecken i texten, utan att de ska tolkas som specialtecken, behöver du \enquote*{bryta ut} dem (eng. \emph{escape}). Det görs med \emph{escape}-tecknet \texttt{\textbackslash} (bakåt snedstreck, eng. backslash). 

          Exempel på specialtecken är \texttt{\textbackslash\%} för procenttecken och \texttt{\textbackslash\$} för dollartecken. \texttt{\textbackslash}-tecknet i sig är också ett specialtecken och kan skrivas med kommandot \texttt{\textbackslash{textbackslash}}, eller \texttt{\textbackslash{backslash}} inom matematikläget.
\end{itemize}

\subsection{Lab 4 --- Maskinkod}
\begin{itemize}
    \item \texttt{Vad betyder svårighetsgraden för uppgifterna?}\\
          Svårighetsgraden är en uppskattning av hur svår varje uppgift är, på en skala från \diffinline{1} (enkel) till \diffinline{5} (mer komplicerad). Den påverkar inte om en uppgift är obligatorisk eller inte -- det framgår alltid av uppgiftsindelningen. 

          Uppgifter under \emph{Datorarbete} är obligatoriska, medan uppgifter under \emph{Extrauppgifter} är frivilliga. Där kan svårighetsgraden vara ett stöd för dig när du väljer vilka uppgifter du vill försöka med.
\end{itemize}



\newpage
\begin{center}
    \vspace*{1cm}
    \Large\textbf{Tack!}
    \vspace{0.5cm}
\end{center}

\noindent
Tack till alla er som tagit er tid att ge feedback på denna laborationhandledning. Era synpunkter har varit till stor hjälp och är mycket uppskattade.

\blankline

\noindent
Om du vill hjälpa till att förbättra denna handledning, eller någonting annat kursrelaterat, kan du antingen kontakt mig direkt (\texttt{\#mattias.nordahl} på Discord eller \url{mattias.nordahl@cs.lth.se}) eller gå till kursens GitHub-repo och skapa en \textit{issue} eller en \textit{pull request}.

\halfblankline

\url{https://github.com/lunduniversity/introprog-computer-intro}

\blankline

\noindent
Nedan följer en lista på personer som har bidragit med feedback:

\begin{itemize}[label={},itemsep=1mm,parsep=0mm]
    \item Björn Regnell
    \item Johan Ekberg
    \item Maximilian Waldenfeldt Uggla
\end{itemize}

\vspace{0.5cm}
\begin{flushright}
    Med vänliga hälsningar,\\
    Mattias Nordahl
\end{flushright}

\newpage
\section{Laboration \arabic{section} --- Linux/Unix}
\label{lab:unix}
\emph{Mål:} Du ska bekanta dig med att använda LTHs Linuxdatorer. Du blir inte expert på Linux (eller Unix) på en laboration, och det behöver du inte heller vara. Men det är viktigt att du är van att arbeta med Linux (och därmed även Unix); det kommer att underlätta dina studier i fortsättningen.


\subsection*{Obligatoriska förberedelser (hemarbete)}
\begin{Hemarbete}
	\item Läs igenom \emph{Appendix B Terminalfönster} i \emph{Introduktion till programmering med Scala} av Björn Regnell.
	\item Läs igenom kompendiet  \emph{Introduktion till LTH:s Linuxdatorer}. Kompendiet är tämligen långt, så börja i god tid. Laborationen innefattar ungefär det som finns i kapitel 1 (Grunderna) och 2 (Påbyggnad). De resterande kapitlen kan du skumma igenom  någorlunda kvickt, för att få en ungefärlig uppfattning av innehållet.
	% Normala år skulle kompendiet fungera som referensmaterial för en introducerande ''datorstuga'' i introduktionsveckan, men höstterminen 2020 är denna inställd på grund av COVID-19. Därför får vi i år nöja oss med att läsa kompendiet som en förberedelse till laborationen.
	\item Kompendiet \emph{Introduktion till LTH:s Linuxdatorer} får användas under laborationen, och hänvisas till i uppgifterna. I hänvisningarna förkortas namnet på  kompendiet till ILL.
\end{Hemarbete}
Länkar till det ovan refererade materialet finns på kurshemsidan under  ''Datorlaborationer'' om du inte har pappersversionerna av dem.


\subsection*{Kontrollfrågor}
\begin{Kontrollfragor}
	\item Hur ser kommandot ut som loggar in på en annan dator via nätet (t.ex. om du vill logga in hemifrån)?
	\item Hur gör man för att byta lösenord på studentdatorerna?
	\item Vad betyder kommandot \code{pwd}?
	\item Vad är en kommandotolk?
	\item Hur får man tillbaka ett tidigare givet kommando, så man kan köra det igen?
	\item Hur kan man få hjälp om användningen av ett kommando, förutsatt att man vet namnet på kommandot?
	\item Hur skriver man ut en innehållsförteckning över den aktuella katalogen?
	\item Vilka är kommandona för att skapa respektive ta bort kataloger?
	\item Hur byter man aktuell katalog?
	\item Vad betyder tecknen \code{?} och \code{*} när man skriver dem på en kommandorad?
	\item Vad betyder tecknen \code{<} och \code{>} när man skriver dem på en kommandorad?
	\item Förklara kort hur systemet med åtkomsträttigheter av filer och kataloger fungerar.
	\item Vilket kommando utnyttjar man om man vill titta på innehållet i en fil en sida i taget?
	\item Antag att programmet \code{prog} producerar många sidor utskrift. Hur gör man för att titta på utskriften en sida i taget?
	\item Vad är en process?
	\item Hur avbryter man ett exekverande program?
\end{Kontrollfragor}

\clearpage
\subsection*{Datorarbete}
Kom ihåg att laborationen är ett inlärningsmoment. Ta hjälp av materialet och labbledaren, och anteckna gärna frågor som du vill diskutera med handledaren under din redovisning.
\begin{Datorarbete}
	\item Logga in på datorn. Använd det användarnamn och det lösenord som du tidigare har kvitterat ut (ILL 1.3).
	\item Fönsterhantering (ILL 1.4). Ägna några minuter åt att bekanta dig med fönstermiljön. Klicka runt bland menyer och applikationer.
	\item Editering av kommandoraden och enkla Unix-kommandon.

	\begin{Deluppgifter}
		\item Öppna ett kommandofönster (Terminal).
		\item Skriv några enkla Unix-kommandon, till exempel \code{pwd}, \code{ls}, \code{date} och \code{cal}.
		\item Skriv avsiktligt fel och rätta felet. Prova specialtecknen för att radera enstaka tecken på kommandoraden och för att radera hela raden (ILL 3.3).
		\item Skriv ut hjälptexten (''man-sidan'') för \code{date}-kommandot (ILL 1.8). Bläddra framåt och bakåt i texten.
		\item Prova specialtecknen för att få tillbaka tidigare kommandon. Återkalla t.ex. kommandot för utskrift av datum och utför det på nytt. Prova både piltangenterna $\uparrow$ $\downarrow$ och \commandchar{control-r}. Du kan också prova kommandot \code{history} för att se vilka kommandon du har kört.
		\item Använd kalenderprogrammet (\code{cal}) för att ta reda på vilken veckodag du är född.
		\item Intresserade kan prova \code{!} (s.k. \emph{history expansion}) för att repetera tidigare kommandon. Exempelvis:
		\begin{Code}
			ls -l      // lista filer, med mer info
			!!         // repetera det senaste kommandot
			man touch  // visa manualen för touch-kommandot
			touch fil  // skapa en fil som heter 'fil'
			!ls        // repetera det senaste kommandot som börjar med 'ls'
		\end{Code}
	\end{Deluppgifter}

	\item Kommandon för att hantera filer och kataloger (ILL 1.7, 2.2, 2.6--2.9).

	\begin{Deluppgifter}
		\item Skriv ut en innehållsförteckning över din hemkatalog. Skriv ut en förteckning där också filer vars namn börjar med punkt (s.k. \emph{punktfiler}) skrivs ut.
		\item Gå till katalogen \code{/usr/local/cs/dod/me/metool/src/metool}. Skriv ut en innehållsförteckning över katalogen. Skriv ut en förteckning över de filer vars namn innehåller strängen \code{Statement}.
		\item Prova hur filnamnskomplettering fungerar. Skriv \code{less R} och tryck på \commandchar{tab}. Datorn fyller i tecken i filnamnet så länge de är unika (nu står det \code{less Re} på kommando\-raden). Det finns mer än en fil vars namn börjar med \file{Re}. Tryck på \commandchar{tab} en gång till (ibland behövs det två extra tryckningar) så får du en lista över dessa filer. Skriv \code{a} och tryck på \commandchar{tab} igen; datorn fyller i till det unika filnamnet \file{ReadStatement.java}. Tryck på \commandchar{return} för att titta på filen.
		\item Gå till din hemkatalog och skapa en katalog för de filer som används i denna laboration. Katalogen ska heta \file{lab1} och vara en underkatalog till en katalog \file{dod}, där du kan spara allt som rör kursen Datorer och datoranvändning. Du kan i fortsättningen skapa en ny katalog för varje datorlaboration som du gör. Använd följande kommandon:

		\begin{Code}
			cd          // gå till hemkatalogen om du inte redan är där
			mkdir dod   // skapa katlogen dod i din hemkatalog
			cd dod      // gå till katalogen dod
			mkdir lab1  // skapa katalogen lab1
			cd lab1     // gå till katalogen lab1
		\end{Code}


		\item Kopiera filen \file{/usr/local/cs/dod/lab1/example.txt} till katalogen \file{lab1}. Skriv ut filen på skärmen med en sida i taget.
		\item Undersök hur mycket utrymme du har tillgängligt för att lagra filer.
		\item Tag reda på hur stor filen \file{example.txt} är. Komprimera därefter filen och tag reda på storleken hos den komprimerade filen. Återställ sedan filen till sitt ursprungliga utseende.
		\item \label{del:h} Skriv ut de rader i filen \file{example.txt} som innehåller ordet Unix. Kommandot för att leta i en fil heter \code{grep}.
		\item \label{del:i} Samma som uppgift \ref{del:h}, men koppla om utskriften så att den hamnar i en fil med namnet \file{unix.txt}. Skriv ut denna fil på skärmen.
		\item Räkna (med ett kommando) antalet rader i filen \file{unix.txt}. Du har nu räknat antalet rader som innehåller ordet Unix i filen \file{example.txt}.
		\item \label{del:k} Tag bort filen \file{unix.txt}.
		\item Gör samma sak som i uppgift \ref{del:i}--\ref{del:k} utan att använda en temporär fil. Koppla i stället ihop kommandona med en pipe (|).
	\end{Deluppgifter}
	\item Editering av text. På LTHs Linuxdatorer finns flera editorer, till exempel \code{nano} (enkel, terminalbaserad), \code{gedit} (enkel, fönsterbaserad), \code{code} (enkel, fönsterbaserad) och \code{emacs} (avancerad). Du får naturligtvis använda vilken editor du vill normalt, men här ska du testa \code{gedit}. Gör gärna om uppgifterna nedan i någon annan editor senare, på egen hand. Vissa uppgifter kommer framstå som väldigt enkla, men prova gärna att göra dem t.ex. i \code{nano}.

	\begin{Deluppgifter}
		\item Starta \code{gedit} och läs in filen \file{example.txt} genom att i terminalfönstret skriva:\\
		\code{gedit example.txt \&}
		\item Utnyttja musen och piltangenterna för att flytta textmarkören. Ändra textinnehållet genom att ta bort tecken och skriva in tecken. Spara det ändrade innehållet till filen.
		\item Kontrollera att filen \file{example.txt} har ändrats.
		\item Dela en rad i två rader. Sätt ihop raden igen.
		\item Lägg in några tomma rader, tag sedan bort dem igen.
		\item Utnyttja rullningslisten för att flytta dig i texten. Gå till början av texten. Gå till slutet av texten. Gå till rad 43 i texten.
		\item Markera ett textblock genom att trycka på vänster musknapp och dra markören.
		\item Markera ett textblock genom att först flytta markören till början av blocket (med musen eller tangentbordet), sedan håll ned Skift-tangenten, och flytta markören till slutet av textblocket (med musen eller tangentbordet).
		\item Experimentera gärna också med tangenterna \texttt{Home} och \texttt{End}, och med \texttt{Ctrl} och piltangenterna \texttt{$\leftarrow$ $\rightarrow$} för att flytta markören och markera text.
		\item Kopiera ett markerat textblock till en annan plats i filen. Flytta sedan ett markerat textblock.
		\item Gå till början av filen och leta upp den första förekomsten av ordet Unix. Leta sedan upp nästa förekomst, osv. Byt sedan alla Unix mot Xinu.
	\end{Deluppgifter}

	\item Hantering av processer (ILL 3.3, 3.8).

	\begin{Deluppgifter}
		\item Skriv kommandot \code{xeyes} i terminalfönstret. Flytta musen så ser du att ögonen följer musmarkören. Notera att man inte kan fortsätta att skriva kommandon i fönstret eftersom det är låst av \code{xeyes}-programmet.
		\item Skriv \commandchar{control-c} i kommandofönstret för att avbryta \code{xeyes}-programmet. Programmets fönster försvinner när man avbryter programmet.
		\item Skriv nu \code{xeyes \&} i terminalfönstret. \code{\&}-tecknet betyder att programmet ska köras som en självständig process som inte är kopplad till terminalfönstret. Nu kan man alltså fortsätta att skriva kommandon i terminalfönstret. Avsluta \code{xeyes} genom att högerklicka med musen på ikonen för \code{xeyes} i verktygsraden på skärmens vänstra sida. Välj \code{Quit} i menyn som visas.
		\item Man kan tillfälligt avbryta exekveringen av ett program med \commandchar{control-z}. Skriv \code{xeyes} och sedan \commandchar{control-z}. Notera att programmet nu inte är aktivt (ögonen följer inte musmarkören). Med kommandot \code{fg} (foreground) återupptar man exekveringen igen. Om man i stället ger kommandot \code{bg} (background) återupptar man exekveringen ``i bakgrunden'', precis som om man hade startat programmet med \code{xeyes \&}.
	\end{Deluppgifter}

	% \clearpage
	\item Inloggning på andra datorer (ILL 3.7).

	\begin{Deluppgifter}
		\item Prova att logga in på datorn \code{login.student.lth.se} med hjälp av kommandot \code{ssh}. Prova att ge några kommandon, t.ex. \code{touch} för att skapa en ny fil. Avsluta genom att skriva \code{exit}.
		\item Datorn \code{login.student.lth.se} är på samma nätverk som datorerna i datorsalarna. Med \code{ssh} loggade du in på en annan dator, men fortfarande med ditt egna konto. Om du skapade en ny fil så kommer du fortfarande hitta den i din lokala terminal efter att du har avslutat \code{ssh}.
		\item Datorn \code{login.student.lth.se} kan nås externt, t.ex. om du behöver logga in hemifrån.
	\end{Deluppgifter}

	\item Glöm inte att logga ut innan du lämnar datorn!
\end{Datorarbete}

\newpage


\begin{frame}
    \frametitle{Vad är versionshantering?}
    
    motivation, t.ex.: ``Har du någonsin...'', mejlat filer fram och tillbaka, etc.
    
    Det är en form av versionshantering.
    
    Har du någonsin...
    förlorat en fil, skrivit över en fil, etc.
    
    lista fler exempel som är relevanta för studenter...
    
\end{frame}
    \frametitle{Versionshanteringssystem}
    
    \begin{frame}
    Hjälper oss att hantera komplexiteten det innebär att hålla reda på alla filer som hör till ett projekt -- källkod, dokumentation, etc.

    \blankline
    Låter oss gå tillbaka till äldre versioner av filerna om nödvändigt.

    \blankline
    Gör det möjligt för oss att arbeta parallellt med filerna i projektet utan att komma i vägen för varandra.
\end{frame}

\begin{frame}
    \frametitle{Versionshanteringssystem}
    
    \begin{block}{Vad är ett versionshanteringssystem?}
        Programvara som lagrar alla versioner som har existerat under projektets gång i någon form av databas.
    \end{block}
    
\end{frame}


\begin{frame}
    \frametitle{Versionshanteringssystem}

    % todo: add figures!
    
    \begin{block}{Todo}
        \begin{itemize}
        \item Centraliserade system (CVS, SVN)
        \begin{itemize}
            \item En central databas, ett repositorium (repo), håller reda på alla filer.
            \item Utvecklare \textit{checkar ut} en lokal kopia av filerna och arbetar med dem.
            \item När arbetet är klart checkar utvecklaren in filerna igen (kopierar dem till det centrala repot) varvid en ny version av projektet skapas.
        \end{itemize}
        \item Distribuerade system (Git)
        \begin{itemize}
            \item Finns inget centralt repo.
            \item I stället klonar man ett repo (gör en komplett kopia av repot).
            \item Ändringar kan hämtas (\textit{pull}), eller överföras till (\textit{push}), andra repon.
        \end{itemize}
        \end{itemize}
    \end{block}
    
\end{frame}



\newpage

\section{Laboration \arabic{section} --- \LaTeX}

\emph{Mål:} Du ska lära dig grunderna i {\LaTeX} och tillämpa dina kunskaper på ett exempel.


\subsection*{Obligatoriska förberedelser (hemarbete)}
\begin{Hemarbete}\firmlist
	\item Titta igenom föreläsningsbilderna till föreläsningen om \LaTeX.
	\item Läs igenom kompendiet \emph{Att skriva rapporter med \LaTeX}, åtminstone så mycket så att du blir bekant med vad man kan göra med \LaTeX. Du behöver inte försöka memorera alla detaljer.
	\item \label{hem:latexuppg} Med början två sidor fram finns ett exempel på en rapport som är producerad med \LaTeX. Studera rapporten och försök komma på vilka kommandon som behövs för att få texten att se ut som den gör. Markera i rapporten, eller i ett separat textdokument, vilka kommandon du behöver använda för att efterlikna rapporten.
\end{Hemarbete}

\Förberedelser
\begin{Kontrollfragor}
	\item Vad är skillnaden mellan {\LaTeX} och vanliga textredigerare, som MS Word?
	\item Vilka fördelar finns med att använda {\LaTeX} för dokument?
	\item Vad är ett {\LaTeX}-kommando och hur skrivs det i en dokumentfil?
	\item Vad är skillnaden mellan ett kommando och en omgivning (environment)?
	\item Hur skapar man rubriker och underrubriker i {\LaTeX}?
	\item Hur får man fet eller kursiv text i {\LaTeX}?
	\item Vad betyder tecket \$ i {\LaTeX}?
	\item Hur skriver du tecknet \$ i {\LaTeX}, utan att det tolkas som ett kommando?
	\item {\LaTeX} är helt textbaserat. Vad är processen för att producera det slutgiltiga, formaterade dokumentet?
	\item Vad är en paketfil i {\LaTeX} och hur inkluderar man den i dokumentet?
	\item Det finns tre typer av listor i {\LaTeX}: punktlistor, numrerade listor och beskrivningslistor. Hur skapar du var och en av dessa?
	\item Hur numreras figurer och tabeller i {\LaTeX}?
	\item Hur refererar du till en figur eller tabell i texten?

\end{Kontrollfragor}

\newpage

\subsection*{Datorarbete}

\textbf{Notis}: Du får avända valfri \LaTeX-editor, t.ex. VS Code, TexmakerI instruktionerna nedan föreslår vi att ni använder programmet Texmaker för att arbeta med \LaTeX, men ni får lov att använda vilken editor ni vill. Det går också bra att använda onlineverktyg, så som Overleaf.

\begin{Datorarbete}
	\item I mappen med labbfiler som du laddade ned i labb 1 finns också en \file{latex}-katalog, som innehåller allt du behöver för att återskapa rapporten på nästa sida. Den finns här:
	
	\url{https://fileadmin.cs.lth.se/pgk/dod-lab-material.zip}
	
	\begin{itemize}
		\item Filen \file{sort\_scala.tex} är en mall för \LaTeX-dokumentet, men utan innehåll.
		\item I filen \file{oformaterad\_text.txt} finns den råa texten utan formattering.
	\end{itemize}
	\item Starta din föredragna \LaTeX-editor och öppna filen \file{sort\_scala.tex}, som är en mall för \LaTeX-dokumentet, med alla paket inkluderade (se kommentarerna i filen). Kopiera in den råa texten från \file{oformaterad\_text.txt} mellan \verb/\begin{document}/ och \verb/\end{document}/.
	\item Återskapa den färdiga rapporten på nästa sida. Lägg in lämpliga \LaTeX-kommandon i filen så att rapporten får (åtminstone ungefär) samma utseende.
	\begin{itemize}
		\item Arbeta stegvis -- ändra lite, bygg/kompilera och titta på resultatet, ändra lite till, osv.
		\item Notera att vissa saker i den råa texten kan leda till att kompileringen misslyckas, t.ex. specialtecken eller matematiska symboler som \(\cdot\) (multiplikation).
		\item Bilderna som ska inkluderas i dokumentet finns också bland de nedladdade labbfilerna, samt programkoden för kodlistningen i rapporten.
		\item Tänk på att vissa detaljer i den råa texten (t.ex. kapitel-, sid- och figurnummer) behöver tas bort, eftersom de genereras automatiskt av \LaTeX.
	\end{itemize}

    \smallskip

    \noindent\textbf{Tips:}
    \begin{itemize}
        \item Kolla vilka paket som är inkluderade i mallen \file{sort\_scala.tex}.
        \item Ta hjälp av \LaTeX-häftet och/eller online-resurser för att hitta rätt kommandon.
        \item Använd etiketter (\verb/\label{etikett}/) och referenser (\verb/\ref{etikett}/) för att referera till figurer, tabeller och sektioner. \LaTeX\ numrerar automatiskt dessa åt dig. Skriv dem inte manuellt.
        \item Den råa texten kan innehålla radbrytningar och bindestreck. Dessa ska inte ingå i källkoden, utan låt \LaTeX\ sköta radbrytningarna automatiskt där det är lämpligt.
        \item Din lösning behöver inte vara \emph{exakt} likadan som den färdiga rapporten. Om du kör fast, fråga labbhandledaren om hjälp.
        \item I appendix numrerar man kapitel med bokstäver (A, B, C, ...). Det görs med kommandot \verb/\appendix/. Rubriker som kommer efter det kommandot numreras med bokstäver.
		\item Ibland kan det vara bra att kommentera ut delar av koden med \verb/%/ i början av raden, för att enklare kunna felsöka.
    \end{itemize}

	\item Om du har tid: prova sådana möjligheter i \LaTeX\ som du inte har behövt använda tidigare: listor av olika slag, innehållsförteckning, mera avancerade formler, osv.
	\item Kolla gärna på lösningen \emph{efter labben}, som finns på GitHub\footnote{Gå till dod-repot: \url{https://github.com/lunduniversity/introprog-computer-intro}\\
	och titta på: \url{lab-instructions/modules/latex/newex/sort_scala_losning.tex}}, men försök att inte titta på den förrän du har gjort så mycket du kan på egen hand.
\end{Datorarbete}

\includepdf[pages=-]{modules/latex/newex/sort_scala_losning}

\newpage

\section{Laboration \arabic{section} --- Maskinkod}

\emph{Mål:} Att förstå hur en dator fungerar på en grundläggande nivå. Man hör ibland att datorer endast förstår och arbetar med ``ettor och nollor'', men vad betyder det egentligen? I denna laboration kommer vi arbeta med en förenklad datormodell där vi kan inspektera och manipulera minnet och processorn på en låg nivå och se just hur ettor och nollor används för att utföra beräkningar.

\subsection{Kort teori}
Här går vi kortfattat igenom några centrala begrepp som du behöver inför laborationen. Gör också uppgifterna i avsnittet ``Obligatoriska förberedelser (hemarbete)'' \emph{innan} laborationen.

\subsubsection{Processor (CPU, Central Processing Unit) och Register}
CPU:n, eller \emph{processorn}, är hjärnan i en dator och utför alla dess beräkningar. Den läser och exekverar instruktioner från arbetsminnet, hanterar data via sina inbyggda register, och koordinerar övriga komponenter i systemet. \emph{Register} är små, snabba lagringsenheter inuti CPU:n som temporärt lagrar data för snabb tillgång under beräkningarna. De är begränsade i antal men erbjuder extremt snabb dataåtkomst -- upp till 100 gånger snabbare än arbetsminnet. Därför vill man vanligtvis använda registerna så mycket som möjligt, genom att ladda in data från arbetsminnet till registren en gång, använda dem genomgående i beräkningar, och först när beräkningarna är klara skriva tillbaka resultatet till arbetsminnet.

\subsubsection{Arbetsminne (RAM, Random Access Memory)}
Arbetsminnet i en dator fungerar som lagringsplatser som kan innehålla en viss mängd data. Man kan tänka på dem som ``lådor'' som kan fyllas med värden som CPU:n sedan kan läsa eller ändra. Varje låda har en unik adress så att CPU:n direkt kan hämta eller skriva data.
Arbetsminnet används för att temporärt lagra både programkod och de data som programmen arbetar med.
Det är långsammare än processorns register, men kan vara betydligt större. En modern dator har typiskt flera miljarder minnesplatser (mäts i GB, Giga Byte). 
Till skillnad från permanenta lagringsenheter som hårddiskar och SSD-enheter, är arbetsminnet \emph{flyktigt}, vilket betyder att dess innehåll försvinner när datorn stängs av. Det är dock många storleksordningar snabbare än både hårddiskar och SSD-enheter\footnote{Läsning från RAM tar typiskt 10--100ns, från SSD 10--100$\mu$s och från hårddisk flera millisekunder.}.


\subsubsection{Instruktioner och Maskinkod}
Varje CPU har en specifik uppsättning instruktioner, känd som dess instruktionsuppsättning. Dessa instruktioner styr CPU:n till att utföra operationer, så som addition och subtraktion eller dataöverföring mellan minnesplatser och register. Vanligt förekommande instruktioner är \texttt{ADD} (addition), \texttt{MOV} (flytta data), och \texttt{JMP} (hoppa till en annan instruktion). Varje instruktion har en binär representation, alltså en sekvens av ettor och nollor. Detta är datorns grundläggande språk, kallat maskinkod.

\subsubsection{Programräknare}
Programräknaren är ett speciellt register i CPU:n som håller reda på adressen till nästa instruktion som ska hämtas och utföras. Detta gör det möjligt för datorn att veta vilken del av programmet som ska köras härnäst.

\subsubsection{Teckenkodning}
För att representera text och tecken använder datorer kodningsscheman. ASCII (uttalas "aski") är ett sådant schema där varje bokstav eller symbol representeras av ett unikt binärt värde. ASCII-schemat innehåller bara 128 olika tecken och inkluderar det engelska alfabetet, siffror och några specialtecken. UTF-8 är ett modernare schema som kan representera de flesta av världens alla språk och alfabet, men är också mer komplicerat, så i denna laboration kommer vi nöja oss med ASCII.

\subsubsection{\progname}
I labben ska du arbeta med en fiktiv dator, kallad \progname, som emulerar (härmar) hur en riktig dator fungerar i princip, men förenklat.
Den har en 8-bitars processor, med åtta register, samt ett arbetsminne med 256 platser. Varje minnesplats är åtta bitar, vilket betyder att varje plats kan lagra ett heltal mellan 0 och 255.
Du kan direkt editera individuella bitar i minnet och i processorns register, och köra dem som instruktioner. Du kommer alltså skriva och köra program i maskinkod.

\subsubsection{Notation i handledningen}
I handledningen behöver vi ibland kunna referera till register och minnesplatser. Registerna i \progname{} har index 0--7, men refereras typiskt till med sina namn:

\begin{description}[leftmargin=20mm]
    \item[\texttt{R0}--\texttt{R2}] Register 0, 1 och 2 är allmänna register och kan användas fritt.
    \item[\texttt{OP1}, \texttt{OP2}] Register som typiskt används som operanderna i beräkningar.
    \item[\texttt{RES}] Ett register som ofta används för att spara resultatet av en beräkning.
    \item[\texttt{OUT}] Ett register som används för att skriva ut data till konsolen.
    \item[\texttt{PC}] Programräknaren, som håller reda på vilken instruktion som ska köras härnäst.
\end{description}

Platserna i arbetsminnet har index 0--255 och benämns som \texttt{M0}, \texttt{M1}, \texttt{M2}, och så vidare. För att skilja mellan när ett värde används direkt eller som en adress för att referera till ett annat värde, använder vi en asterisk (eller ``stjärna'', \texttt{*}). Denna notation är inspirerad av programspråket C, där asterisk används för att hantera pekare. Som exempel: Om värdet 38 är lagrat i registret \texttt{R0}, så refererar \texttt{R0} till värdet 38. Å andra sidan, betyder \texttt{*R0} att värdet 38 används som en adress, och refererar till \texttt{M38}, alltså värdet som finns i arbetsminnet på plats 38. För mer komplexa adresseringar kan vi även använda parenteser, såsom i \texttt{*(R0+1)}, vilket då refererar till minnesplatsen \texttt{M39}.

\newpage

\subsection*{Obligatoriska förberedelser (hemarbete)}
\begin{Hemarbete}\firmlist
    \item Läs igenom den korta teorin ovan och se till att du förstår begreppen.
    \item Titta igenom föreläsningsbilderna från föreläsning 4.
    \item Läs igenom uppgifterna under rubriken Datorarbete.
    \item Förbered din dator för laborationen genom att installera Java om du inte redan gjort det. Instruktioner finns på kurshemsidan.
    \item Ladda ned programmet \textbf{\progname} (länk nedan) och packa upp det i en lämplig katalog. Programmet består av en \texttt{.jar}-fil som innehåller allt som behövs för att köra programmet.\\
    $\rightarrow$ \url{https://github.com/lunduniversity/introprog-cpu-emulator/releases/latest}
    \item Kontrollera att du kan köra programmet genom att dubbelklicka på \texttt{.jar}-filen. Om det inte fungerar kan du istället starta programmet från terminalen med kommandot:
    \begin{center}
        {\code{java -jar \progfilename}}
    \end{center}
    Där \code{\progversion} är versionsnummret, vilket eventuellt kan vara annorlunda.
    \item Läs igenom användarmanualen för programmet som beskriver hur det fungerar. Manualen öppnas automatiskt när programmet startas, eller via menyn \texttt{Help}~$\rightarrow$~\texttt{Show Help}.
\end{Hemarbete}

\subsection*{Kontrollfrågor}
\begin{Kontrollfragor}
    \item Vad står förkortningen CPU för?
    \item Vad är huvudfunktionen hos en CPU?
    \item Nämn två typer av minne som finns i en dator.
    \item Vad används registren i en CPU till?
    \item Vad är en programräknare och vad gör den?
    \item Vad menas med termen "cache" i kontexten av CPU?
    \item Hur representeras informationen i datorns minne?
    \item Vad innebär binärt tal?
    \item Hur konverterar man ett decimaltal till ett binärt tal?
    \item Vad är en maskininstruktion?
    \item Vad är en assembler?
    \item Hur skiljer sig maskinkod från högnivå programmeringsspråk?
    \item Vad är ett operativsystem och vilken roll spelar det i en dator?
    \item Vad är en bit och vad är en byte?
    \item Vad är ett hexadecimalt tal och varför används det ofta i datavetenskap?
    \item Ge ett exempel på en enkel operation som en ALU (Arithmetic Logic Unit) kan utföra.
\end{Kontrollfragor}

% 

\clearpage
\subsection*{Datorarbete}
Under laborationen kommer vi att arbeta med programmet \progname{} som simulerar en förenklad dator. Programmet är skrivet i Java och kan köras på de flesta datorer. Datorn har ett minne och en processor med register som kan utföra enkla operationer. Programmet låter oss inspektera och manipulera minnet och processorn på en låg nivå och se hur datorn tolkar och utför instruktioner.


\begin{Datorarbete}
    \item \difficulty{1} Börja med att starta \progname. Om du inte gjort det redan så läs igenom användarmanualen som beskriver hur programmet fungerar. Den öppnas automatiskt när programmet startas, eller via menyn \texttt{Help}~$\rightarrow$~\texttt{Show Help}.

    \item \difficulty{1} I programmet finns ett antal exempelprogram som kan laddas in, under menyn \texttt{Examples}. Öppna det exempel som heter \texttt{Tiny program}. Detta är ett mycket enkelt program som bara använder fyra minnesceller, varav tre innehåller instruktioner och en innehåller data. Prova att stega igenom programmet och se vad som händer vid varje steg. Nedan är en beskrivning av vad programmet gör:
    \begin{enumerate}
        \item Eftersom programräknaren startar på värdet 0 kommer processorn att läsa värdet på den minnesplatsen först och tolka det som en instruktion.
        \item Instruktionen är \texttt{LD (72 $\rightarrow$ OUT)}, eller "Load 72 into OUT". \texttt{LD}-instruktionen läser värdet på nästa minnesplats, i detta fall det decimala värdet 72, och laddar in det till ett register. Registret bestäms av intruktionens \emph{operand}, alltså instruktionens fyra sista bitar. Här anges registret med index 6, som är \texttt{OUT}.
        \item När instruktionen körs så läses alltså värdet på minnesplats 1, och kopieras till registret \texttt{OUT}. Därefter ökar programräknaren med 2, eftersom instruktionen är två byte lång.
        \item Nu pekar programräknaren på minnesplats 2, som innehåller instruktionen \texttt{PRT}, eller "Print text". Denna instruktion skriver ut värdet som finns i registret \texttt{OUT} till konsolen, som ett ASCII-tecken. Därefter ökar programräknaren med 1.
        \item Till sist så körs instruktionen på minnesplats 3, som är \texttt{HLT}, eller "Halt", som avslutar programmet.
    \end{enumerate}

    \item \difficulty{2} Bygg vidare på det tidigare exemplet och skriv ut flera tecken i rad, t.ex. \texttt{Hello,~World!}. Kom ihåg att varje tecken måste representeras som ett ASCII-värde. Ta hjälp av ASCII-tabellen för att hitta tecken och deras motsvarande decimal- eller binärvärden.

    När du är klar så spara ditt program genom att klicka på \texttt{File}~$\rightarrow$~\texttt{Save} och ge filen ett namn. Du kan sedan ladda in programmet igen genom att klicka på \texttt{File}~$\rightarrow$~\texttt{Open} och välja filen.
    \hint[1]{Tecknet " " (mellanslag) har ASCII-värdet 32.}
    \vspace{-2mm}
    \hint[2]{Du kan använda upprepade \texttt{LD}- och \texttt{PRT}-instruktioner, eller \texttt{PRL} (Print Loop).}

    \item \difficulty{2} Från \texttt{Examples}-menyn, öppna nu det exempel som heter \texttt{Simple~add}, och försök förstå vad det gör och hur det fungerar. Kom ihåg och ta hjälp av att varje minnescell visas med olika tolkningar, däribland decimalt och som instruktioner. Vilka värden kommer att tolkas som instruktioner och vilka som data? Vad kommer resultatet att bli när programmet körs? Till sists, kör programmet och se om du hade rätt.
    
    % Force a new page, to not split the next assignment
    \newpage

    \item \difficulty{4} Nu ska du modifiera ditt program lite grand, och utöka din förståelse för CPU:n genom att hantera data dynamiskt. Föreställ dig att ett tidigare program har kört och sparat ett resultat i minnet på en okänd plats; alltså du känner inte till minnesplatsen när du skriver programmet, men programmet får reda på adressen under körning.

    \begin{Deluppgifter}
        \item \textbf{Förbered Minnet:} För att simulera den tidigare beräkningen, skriv manuellt in talen 127 och 43 på minnesplatserna 18 respektive 19. Detta är alltså de "okända" värdena som programmet kommer att läsa.
        \item \textbf{Ladda Adresser:} Använd instruktionen \texttt{LD} för att ladda in värdet 18 till ett register, t.ex. \texttt{R0}. Notera att detta värdet alltså ska tolkas som en minnesaddress, och i sin tur användas för att ladda de faktiska värdena. Register \texttt{R0} kommer alltså att innehålla adressen som värdet finns på.
        \item \textbf{Ladda Värden:} Använd instruktionen \texttt{LDA} (Load Address) för att ladda värden från en given address till ett register. Alltså, ladda in operanderna som ska adderas från \texttt{*R0} och \texttt{*(R0+1)} till operandregistren, \texttt{OP1} och \texttt{OP2}.
        \item \textbf{Ändra till Subtraktion:} Modifiera programmet för att utföra en subtraktion istället för addition, alltså beräkna \(127-43\), och skriva ut resultatet liksom tidigare. Kontrollera att programmet ger det korrekta resultatet.
        \item \textbf{Experiment med Operandordning:} Vad tror du händer om du byter plats på värdena i minnesplatserna och beräknar \(43-127\)? Prova och se vad resultatet blir.
    \end{Deluppgifter}

    \noindent När du är klar så spara ditt program i en fil \texttt{subtraction.txt}.

    \hint{För deluppgift c): \texttt{LDA} instruktionen laddar in värdet från en minnesplats, vars address ges av ett register. \texttt{*RO} betyder att värdet i \texttt{R0} används som en adress, och värdet på den adressen laddas in till ett register, t.ex. \texttt{OP1}. \texttt{*(R0+1)} betyder att värdet i \texttt{R0} först ökas med 1, och det nya värdet används som adress för att ladda in ett värde. Du behöver alltså öka värdet i \texttt{R0} manuellt innan du använder det för att ladda in nästa värde.}

    \item \difficulty{3} Skriv ett eget program som innehåller en \texttt{if}-sats. Programmet ska jämföra två tal och skriva ut Y (för "yes") om de är lika, annars N (för "no"). Använd hoppinstruktioner för att implementera villkorlig exekvering. Kör programmet och se att det fungerar som förväntat.
    \hint{Liksom alltid inom programmering så kan målet uppnås på flera olika sätt. Nedan är två exempel i pseudokod:}

    \begin{minipage}[t]{0.42\textwidth}
        \begin{lstlisting}[xleftmargin=-15mm]
            1. a = 5
            2. b = 5
            3. if a == b, jump to 5
            4. jump to 7
            5. load and print 'Y'
            6. halt
            7. load and print 'N'
            8. halt
        \end{lstlisting}
    \end{minipage}
    \begin{minipage}[t]{0.42\textwidth}
        \begin{lstlisting}[xleftmargin=-15mm]
            1. a = 5
            2. b = 5
            3. load 'Y'
            4. if a == b, jump to 6
            5. load 'N'
            6. print
            7. halt
        \end{lstlisting}
    \end{minipage}

    \item \difficulty{2} Nästa exempel är mycket roligt! :) \\
    Öppna exempelprogrammet \texttt{Simple loop} och försök förstå vad det gör och hur det fungerar. Vad kommer resultatet att bli när programmet körs? Kör det och se om du hade rätt.

    % Force a new page, to not split the next assignment
    \newpage

    \item \difficulty{3} Vi kan alltså skapa loopar genom att använda hoppinstruktioner. Skapa ett eget program som räknar från 0 till 10 och skriver ut varje tal. Använd en loop för att uppnå detta. Kör programmet och se att det fungerar som förväntat. Spara programmet i en ny fil \texttt{count.txt}.
    
    \item \difficulty{2} Baserat på ditt \texttt{count}-program, skriv ett nytt loop-program som summerar alla tal från 0 till \(N\), där \(N\) är ett tal som du själv väljer. Eftersom \progname{} endast har en 8-bitars processor så är vi väldigt begränsade i hur stora tal vi kan hantera. Vilken är den största summan du kan beräkna? Vad händer om du beräknar en större summa? Spara ditt program i en ny fil \texttt{sum.txt} när du är klar.
\end{Datorarbete}


\subsection{Extrauppgifter (frivilliga)}

Nedanstående uppgifter krävs inte för att bli godkänd på laborationen, utan finns för dig som vill ha en extra utmaning. Kika gärna på dem och se om någon verkar intressant! I annat fall är du klar med laborationen och kan redovisa den för handledaren.


\begin{Extrauppgifter}
    \item \difficulty{4} Skriv ett program som konverterar en följd av siffror till en sträng av motsvarande tecken. Programmet ska läsa in en följd av värden (som måste vara 0--9) från minnet och konvertera dem till motsvarande ASCII tecken. Alltså, den första följden av siffror ska kunna skrivas till terminalen med \texttt{PRD} instruktionen (print decimal), medan de konverterade värdena ska kunna skrivas ut med \texttt{PRT} instruktionen (print text).

    \item \difficulty{2}Skriv ett program som kontrollerar ifall två tal båda är större än 50, och i så fall skriver ut "True", annars "False". Programmet ska alltså göra \emph{två} jämförelser i ett "och"-uttryck, motsvarande ungefär:
    \begin{Code}
        if a > 50 && b > 50:
            ...
        else:
            ...
    \end{Code}
    \vspace{-3mm}
    \hint{En villkorlig hoppinstruktion som inte uppfyller villkoret kommer fortsätta att exekvera nästa instruktion som vanligt.}

    \item \difficulty{4} Implementera ett funktionsanrop i ditt program. Skapa en funktion som adderar två tal och returnerar summan. Använd hoppinstruktioner för att anropa funktionen och hantera returvärdet. Ditt program ska motsvara följande pseudokod:
    \begin{Code}
        a = 3
        b = 4
        sum = call add(a, b)
        print sum
        halt
        add(a, b):
            return a + b
    \end{Code}
    \vspace{-3mm}
    \hint{Se nedan för förslag på hur en funktion kan implementeras. Notera att termen "funktion" används ganska löst här, eftersom det inte finns något koncept av funktioner i maskinkod.}
    \begin{enumerate}
        \item Välj för din funktion fyra register som ska ha särskild betydelse:
              \begin{itemize}
                  \item Ett register för att spara det första talet (första funktionsparametern).
                  \item Ett register för att spara det andra talet (andra funktionsparametern).
                  \item Ett register för att spara programräknarens nuvarande värde, så att funktionen vet vart den ska hoppa tillbaka.
                  \item Ett register för att spara returvärdet, så att koden som anropade funktionen kan använda det.
              \end{itemize}

              Använd förslagsvis registerna \texttt{R1}--\texttt{R3} för att spara de två talen och programräknarens värde. Returvärdet behövs inte förrän efter beräkningen är klar, så du kan återanvända ett av de två talregistren för att spara det, eller använda \texttt{RES}.
        \item För koden som ska anropa funktionen:
              \begin{enumerate}
                  \item Spara de två talen och programräknarens nuvarande värde i dina valda register.
                  \item Hoppa till funktionens startadress.
              \end{enumerate}
        \item För funktionen:
              \begin{enumerate}
                  \item Hämta de två talen från dina valda register och utför beräkningen.
                  \item Spara resultatet i returregistret.
                  \item Använd det sparade programräknarvärdet för att hoppa tillbaka till den instruktion som anropade funktionen.
              \end{enumerate}
        \item Tänk på att funktionen ska retunera och fortsätta med instruktionen \emph{efter} anropet, så att programmet inte hamnar i en oändlig loop.

    \end{enumerate}

    \item \difficulty{4} Skriv ett program som multiplicerar två tal genom upprepade additioner. Använd en loop för att utföra multiplikationen och spara resultatet i en minnescell.

    \item \difficulty{4} När en funktion anropas är det viktigt att inte överskriva nuvarande värden i vissa register som används av programmet. I verkliga processorer finns det därför speciella register kända som \emph{mottagarbevarade} (eng. \emph{callee-saved}), vilka ska sparas och återställas av den anropade funktionen. Implementera en funktion som använder och tillfälligt sparar registren \texttt{R1} till \texttt{R3} och återställer dem innan funktionen avslutas.

    \hint{Använd ett ledigt område i minnet (i verkliga system skulle detta hanteras av operativsystemet) för att temporärt spara de register som ska bevaras. Nu kan funktionen fritt använda registerna till sina beräkningar. Innan funktionen avslutas och returnerar, återställ dessa register till deras ursprungliga värden.}

    \item \difficulty[Väldigt svår. Lösningsförslag i Examples-menyn, \texttt{Segfault}.]{5}
    I programspråket C är \emph{segmentation fault} ett ökänt och vanligt förekommande fel. Det betyder att programmet försöker skriva till eller läsa från en minnesadress som inte är tillåten (t.ex. en adress utanför programmet eller en del av minnet som inte är allokerat).    
    Implementera ett program som medvetet leder till segmentation fault, t.ex. som demonstrerat av nedanstående pseudokod. Reflektera sedan över vad som händer och varför det kan så vara svårt att felsöka denna typ av fel.

    \halfblankline
    \textbf{OBS!} Tänk på att spara ditt program \emph{innan} du kör det, gärna en kopia också, eftersom när du får det att ``fungera'' så kommer det att skriva över sig själv och till slut krascha. Spara alltså \emph{inte} programmet efter att det skrivit över sig själv!

    \hint[1]{Pseudokoden nedan använder en mix av hög- och lågnivåinstruktioner, för att både visa det logiska programflödet men samtidigt ge ledtrådar till kluriga lågnivådetaljer. Din kod kommer behöva använda fler rader och instruktioner, och du behöver tänka på hur minnesadresser och register används, så radnummer kommer inte stämmer.}

    \hint[2]{Förklaring: Tanken med programmet är att det får ett startvärde (\(val = 10\)) och ett maxvärde (\(max = 200\)). Startvärdet dubbleras så länge det är mindre än maxvärdet, och varje beräknat värde sparas också i en vektor (array). Vi vill alltså beräkna vektorn \(nums = [20, 40, 80, 160, 320]\). Med andra startvärden blir vektorn annorlunda. Om startvärdet är $1$ så blir vektorn $nums = [2, 4, 8, 16, 32, 64, 128, 256]$, där antalet element är $|nums| = 8$. I värstafallet behöver vektorn alltså ha plats för åtta värden.}

 
    \begin{Code}
        0. jump to 12  // allocate memory for variables
        1. 10          // val = 10
        2. 200         // max = 200
        3. 0           // idx = 0
        4. 0           // nums = int[8], i.e. 8 empty memory cells
        (*@\dots@*)
       11. 0
       12. if val > max, jump to 17:   // loop until val >= max
       13.     val = val + val
       14.     store val in nums[idx]  // i.e. in m[4 + idx]
       15.     idx = idx + 1
       16.     jump to 12
       17. halt                        // do something with nums...
                                       // for this example,
                                       // just stop the program
    \end{Code}

\end{Extrauppgifter}


% \newpage

% \subsection*{Datorer och datoranvändning, godkända laborationsuppgifter}
% \sectionmark{Godkända laborationsuppgifter}

% Efter du blivit godkänd på en laboration kan du be din handledare att signera nedan. Detta är inget som krävs, för handledaren kommer anteckna ditt godkännande och föra in i vårt betygsystem oavsett vilket. Däremot kan det fungera som en liten försäkring för din egen skull; Om det trots allt skulle ske ett misstag någonstans och ditt godkännande inte förs in i betygsystemet, så kan du visa att du blivit godkänd med nedanstående underskrift.

% \vspace{3mm}

% \noindent Värt att notera är att de flesta kurser du kommer läsa på LTH inte har något liknande analogt system för att visa att du blivit godkänd på en laboration, utan sköter det helt digitalt.

% \vspace{3mm}

% \noindent Skriv ditt namn och din namnteckning nedan:

% \blankline
% \blankline
% \n Namn: \dotfill\\

% \blankline
% \n Namnteckning: \dotfill\\

% \blankline
% \begin{tabular}{lcc}
% 	\toprule \addlinespace
% 	{\sffamily\small Godkänd laboration	} & {\sffamily\small Datum} & {\sffamily\small Laborationsledarens namnteckning} \\ \addlinespace \midrule
% 	1 Linux                                                                                                              \\ \addlinespace \midrule
% 	2 \LaTeX                                                                                                             \\ \addlinespace \midrule
% 	3 Git                                                                                                                \\ \addlinespace \midrule
% 	4 Maskinkod                                                                                                          \\ \addlinespace
% 	\bottomrule
% \end{tabular}

\end{document}

\title{Unix/Linux}
\section{Unix/Linux}

\begin{frame}
    \label{unix}
    \begin{block}{\centering\Large Föreläsning \arabic{section} --- Unix/Linux}
        Förberedelse inför laboration 1.

        \begin{itemize}
            \ii{Operativsystem, Unix historik}
            \is{Filer och kataloger}
            \is{Kommandon}
            \is{Filskydd}
            \is{Kommandotolk}
            \is{Processer}
        \end{itemize}
    \end{block}
\end{frame}


\begin{frame}[fragile]
    \frametitle{Operativsystem}
    \begin{itemize}
        \ii{Dator --- kör program}
        \ii{Program --- instruktioner för datorn}
        \ii{Operativsystem --- en samling program som gör det möjligt att köra \enquote{vanliga} program}
        \ii{Operativsystemet hanterar:}
        \begin{itemize}
            \is{de program som körs (program körs ofta parallellt, operativsystemet ser till att programmen får minne och tid att exekvera)}
            \ii{yttre enheter (tangentbord, mus, skärm, nätverk, \ldots)}
            \ii{lagring av data (filer, \ldots)}
            \ii{skydd, felhantering, kommunikation med användaren}
        \end{itemize}

        \ii{Exempel:}
        \begin{itemize}
            \is{Linux, Windows, Mac OS X, Unix, \ldots}
        \end{itemize}
    \end{itemize}
\end{frame}

\begin{frame}[fragile]
    \frametitle{Unix historia}
    \begin{itemize}
        \ii{1960-talet: Multics, ett stort projekt avsedd för stora datorer.}
        \ii{1969: Ken Thompson på AT\&T Bell Labs utvecklar Unix, ett enklare system för mindre datorer med inspiration från Multics.}
        \ii{1973: Unix omskrivs i C av Dennis Ritchie, vilket underlättar distribution och anpassning.}
        \ii{Utvecklingen fortsätter med många varianter: Linux, BSD, Mac OS X, och Android bland andra.}
        \ii{Unix varumärket ägs av The Open Group som certifierar Unix-kompatibla system (Linux är ej certifierat).}
    \end{itemize}
\end{frame}

\begin{frame}[fragile]
    \frametitle{Linux och GNU}
    \begin{itemize}
        \ii{1983: GNU-verktygen -- fritt Unix-liknande operativsystem.}
        \ii{1991: Linus Torvalds skapar Linuxkärnan -- hanterar hårdvaruinteraktioner.}
        \ii{Linuxkärnan + GNU-verktyg = GNU/Linux -- skapar en fullständig användarmiljö.}
        \ii{\enquote{Linux} används ofta som kortform för hela systemet.}
    \end{itemize}
\end{frame}


\begin{frame}[fragile]
    \frametitle{Kännetecken för Unix}
    \ti{Grundläggande:}
    \begin{itemize}
        \is{Kärna --- liten, grundläggande funktioner}
        \ii{Processer, tidsdelning}
        \ii{Filskydd}
        \ii{Fleranvändarsystem}
        \ii{Kommandotolk (terminalfönster)}
    \end{itemize}
    \ti{Unix är:}
    \begin{itemize}
        \is{Litet, standardiserat, flyttbart}
        \ii{(Ursprungligen) Inte för nybörjare. Enklare nu med grafiska skrivbordsmiljöer.}
    \end{itemize}
\end{frame}

\begin{frame}[fragile]
    \frametitle{Vilket operativsystem använder du?}

    \begin{figure}
        \centering
        \includegraphics[height=0.5\textheight]{images/Mentimeter_Branding_Logo.png}
    \end{figure}

\end{frame}

\begin{frame}[fragile]
    \frametitle{Filer i Unix/Linux}
    I en dator måste man kunna lagra
    \begin{itemize}
        \ii{program (Word, Excel, Java-/Scalakompilator, spelprogram, \ldots)}
        \ii{data (foton, document, programmeringsuppgifter, \ldots)}
    \end{itemize}

    \halfblankline

    \ti{Man lagrar program och data i \emph{filer}, som har ett namn.}

    \halfblankline
    \ti{Filer ligger alltid i en katalog (mapp).}

    \halfblankline
    \ti{Filer och kataloger lagras på \enquote{permanent} minne (hårddisk, SSD, \ldots).}

    \halfblankline
    \ti{Filnamn i Unix/Linux:}

    \begin{itemize}
        \is{\code{namn.tillägg} (\code{Calc.scala}, \code{Calc.class}, \code{brev.txt}, \code{test1}, \ldots)}
        \ii{Tillägget är bara en extra upplysning, bestämmer inte filtypen!}
    \end{itemize}
\end{frame}

\begin{frame}[fragile]
    \frametitle{Kataloger}
    Varje fil finns i en katalog. Kataloger kan innehålla underkataloger, så man får en hierarkisk struktur, ett \emph{filträd}:

    \begin{center}
        % \includegraphics[width=\textwidth,height=0.7\textheight,keepaspectratio]{images/file-tree.pdf}
        \resizebox{!}{0.7\textheight}{\documentclass{standalone}
\usepackage{forest}

\begin{document}

\begin{forest}
    for tree={
    font=\ttfamily,
    grow=south, % Tree grows down
    s sep=8mm, % Horizontal space between siblings
    l sep=6mm, % Vertical space between parent and child
    child anchor=north,
    parent anchor=south,
    anchor=north,
    align=center,
    % edge path={
    %   \noexpand\path [draw, \forestoption{edge}]
    %   (!u.parent anchor) -- ([yshift=2mm].child anchor) -| (.child anchor) \forestoption{edge label};
    % },
    edge path={
            \noexpand\path [draw, \forestoption{edge}]
            (!u.parent anchor) |- ([yshift=2mm].child anchor) -| (.child anchor) \forestoption{edge label};
        },
    draw, % Draw the border of nodes
    rounded corners=3mm, % Round corners of the nodes
    fill=white, % Fill color of nodes
    % drop shadow, % Optional: adds a shadow to nodes
    minimum height=6mm, % Minimum height of nodes
    minimum width=10mm, % Minimum width of nodes
    inner xsep=2mm, % X Padding within nodes
    inner ysep=1mm, % Y Padding within nodes
    calign=center
    }
    [/
    [var]
    [usr]
    [h
        [\(\ldots\),draw=none]
        [d23]
        [d24
            [a]
            [b
                [\(\ldots\),draw=none]
                [ma1337no-s]
                [uw3307bq-s
                    [Downloads]
                    [pgk
                        [lab1
                            [kojo.scala,draw=none]
                        ]
                        [lab2]
                        [\(\ldots\),draw=none]
                    ]
                    [dod
                        [lab1
                            [\(\ldots\),draw=none]
                            [example.txt,draw=none]
                            [\(\ldots\),draw=none]
                        ]
                        [lab2]
                        [lab3]
                        [lab4]
                    ]
                    [\(\ldots\),draw=none]
                ]
                [\(\ldots\),draw=none]
            ]
            [c]
            [d]
            [\(\ldots\),draw=none]
        ]
        [\(\ldots\),draw=none]
    ]
    [bin]
    [etc]
    ]
\end{forest}

\end{document}
}
    \end{center}

\end{frame}

% \begin{frame}[fragile]
%     \frametitle{Filnamn}
%     I praktiken är det inte så rörigt som det ser ut på den förra bilden!
%     Ett fullständigt (absolut) filnamn börjar från rotkatalogen och man räknar upp alla katalogerna med \code{/} emellan:

%     \begin{Code}
%         /h/d24/b/uw3307bq-s/dod/lab1/example.txt
%     \end{Code}

%     Men operativsystemet håller reda på en \emph{aktuell katalog}, som ungefär betyder \emph{\enquote{var i filträdet befinner jag mig just nu?}}. Istället för att ange det absoluta filnamnet, räcker det att ange vägen med start från den aktuella katalogen -- ett s.k. \emph{relativt} filnamn.

%     \halfblankline

%     Om \code{uw3307bq-s} (användarens \emph{hemkatalog}) är aktuell katalog så kommer man åt filen ovan med namnet:

%     \begin{Code}
%         dod/lab1/example.txt
%     \end{Code}

% \end{frame}

\begin{frame}[fragile,t]
    \frametitle{Filnamn och sökvägar}

    \ti{\textbf{Termer:}}
    \begin{itemize}
        \is{Sökväg (eng. path) -- vägen till en fil}
        \ii{Absolut sökväg -- från rotkatalogen}
        \ii{Relativ sökväg -- från aktuell katalog}
        \ii{Aktuell katalog -- \emph{där du är i filträdet}}
    \end{itemize}

    \blankline
    \ti{\textbf{Absolut filnamn:}}
    \begin{onlyenv}<.->
        \begin{GobbleCode}{6}
            /h/d24/b/uw3307bq-s/dod/lab1/example.txt
        \end{GobbleCode}
    \end{onlyenv}

    \ti{\textbf{Relativt filnamn:}}
    \begin{onlyenv}<.->
        \begin{GobbleCode}{6}
            dod/lab1/example.txt
        \end{GobbleCode}
    \end{onlyenv}

\end{frame}


% \begin{frame}[fragile]
%     \frametitle{Förkortade filnamn}
%     En del filer har förkortade namn:

%     \blankline
%     \begin{tabular}{cl}
%         \url{~}     & (tilde) hemkatalogen för aktuell användare (Alt Gr + $\sim$ ) \\
%         \url{~user} & hemkatalogen för användaren med användarnamnet user           \\
%         \code{.}    & (punkt) aktuell katalog                                       \\
%         \code{..}   & (punktpunkt) katalogen ovanför aktuell katalog i filträdet    \\
%     \end{tabular}

%     \blankline
%     Antag att \code{\url{~}/dod/lab1} är aktuell katalog. Man kan navigera upp och ned i filträdet enligt följande exempel:

%     \begin{GobbleCode}{6}
%         ~/pgk/lab1/kojo.scala
%         (till hemkatalogen, ner i pfk, ner i lab1)

%         ../../pgk/lab1/kojo.scala
%         (upp ett steg, upp ett steg, ner i pgk, ner i lab1)
%     \end{GobbleCode}

% \end{frame}

\begin{frame}[fragile,t]
    \frametitle{Förkortade filnamn}

    \ti{\textbf{Förkortningar:}}

    \ts{Vissa filnamn har speciella betydelser:}

    \blankline
    \begin{onlyenv}<.->
        \begin{tabular}{cl}
            \url{~}     & Hemkatalog (tilde)              \\
            \url{~user} & Användarens hemkatalog          \\
            \code{.}    & Aktuell katalog (punkt)         \\
            \code{..}   & Överordnad katalog (punktpunkt) \\
        \end{tabular}
    \end{onlyenv}
    \blankline

    \ti{\textbf{Exempel:}}

    \ts{Om \code{\url{~}/dod/lab1} är aktuell katalog:}

    \begin{onlyenv}<+->
        \begin{GobbleCode}{10}
            ~/pgk/lab1/kojo.scala
            (till hemkatalogen, ner i pfk, ner i lab1)
        \end{GobbleCode}
    \end{onlyenv}
    \vspace{-1em}
    \begin{onlyenv}<+->
        \begin{GobbleCode}{10}
            ../../pgk/lab1/kojo.scala
            (upp ett steg, upp ett steg, ner i pgk, ner i lab1)
        \end{GobbleCode}
    \end{onlyenv}

\end{frame}


% \begin{frame}[fragile]
%     \frametitle{Kommandotolk}
%     Man kommunicerar med Unix eller Linux genom att ge kommandon. Ett särskilt program i operativsystemet, kommandotolken, ''shell'', läser in kommandona och utför det som ska göras.

%     \pindent Det finns flera olika kommandotolkar. På Linuxdatorerna i E-huset används kommandotolken \code{bash} (''Bourne-Again SHell''). Andra vanliga kommandotolkar är \code{sh}, \code{zsh} och \code{csh}.

%     \pindent Kommandotolken läser kommandon som ges i ett terminalfönster på skärmen. Så här arbetar tolken:

%     \begin{GobbleCode}{4}
%         Upprepa i all oändlighet:
%             Läs ett kommando
%             Om kommandot är ett riktigt kommando:
%                 utför det som ska göras
%             annars:
%                 skriv ut felmeddelande
%     \end{GobbleCode}

% \end{frame}

\begin{frame}[fragile]
    \frametitle{Kommandotolk (shell) och terminal}

    \ti{Termerna \emph{kommandotolk} och \emph{terminal} används ofta synonymt, men har olika betydelser:}

    \halfblankline
    \begin{description}
        \di{Kommandotolk}{För kommunikation med operativsystemet, via kommandon}
        \di{Terminal}{Program som låter användaren skriva kommandon till kommandotolken}
        \di{Shell}{Engelska för kommandotolk}
    \end{description}

    \blankline

    \ti{Exempel på shells:}
    \begin{itemize}
        \is{\code{bash} (Bourne-Again SHell)}
        \is{\code{sh}, \code{zsh}, \code{csh}}
    \end{itemize}

\end{frame}

\begin{frame}[fragile]
    \frametitle{Kommandotolk (shell) och terminal}

    Så här fungerar en kommandotolk:
    \begin{GobbleCode}{4}
        Upprepa i all oändlighet:
        Läs ett kommando
        Om kommandot är ett riktigt kommando:
        utför det som ska göras
        annars:
        skriv ut felmeddelande
    \end{GobbleCode}

\end{frame}

\begin{frame}[fragile]
    \frametitle{Bash vs. Zsh för Mac-användare}

    \ti{\textbf{Bash och Zsh:}}
    \begin{itemize}
        \is{Bash är standard i de flesta Linux-system.}
        \is{Zsh är standard i macOS från Catalina.}
        \is{Båda är kraftfulla skal i Unix-system.}
        \is{För grundläggande användning, t.ex. denna kurs, fungerar båda bra.}
    \end{itemize}

    \ti{\textbf{Att byta till Bash (valfritt):}}
    \begin{itemize}
        \is{Öppna Terminal och skriv \code{bash} för att temporärt byta till Bash.}
        \is{Installera senaste Bash via Homebrew: \code{brew install bash}}
    \end{itemize}

    \ti{\textbf{Skillnader:}}
    \begin{itemize}
        \is{Zsh upplevs av många som mer användarvänligt. Erbjuder fler funktioner och anpassningsmöjligheter, men kan vara överväldigande för nybörjare.}
        \is{Bash är mer konservativt och har bättre kompatibilitet.}
    \end{itemize}

\end{frame}

\begin{frame}[fragile]
    \frametitle{Unix-shellalternativ för Windows-användare}

    \ti{\textbf{Git Bash:}}
    \begin{itemize}
        \is{Enkel Bash-tillgång, minimal konfiguration.}
        \is{Installeras med Git för Windows.}
    \end{itemize}

    \ti{\textbf{WSL (Windows Subsystem for Linux):}}
    \begin{itemize}
        \is{Kör fullständig Linux i Windows.}
    \end{itemize}

    \ti{\textbf{Cygwin:}}
    \begin{itemize}
        \is{Ger bred Unix-vertygsuppsättning.}
        \is{Mer konfiguration.}
    \end{itemize}

    \blankline
    \ti{\textbf{Vilket ska jag välja?}}
    \begin{itemize}
        \is{\textbf{Git Bash}: Grundläggande Bash och Git. (Rekommenderas)}
        \is{\textbf{WSL}: Fullständig Linux-utveckling. (Rekommenderas)}
        \is{\textbf{Cygwin}: Omfattande Unix-funktionalitet. (Avråds)}
    \end{itemize}

\end{frame}


\begin{frame}[fragile,t]
    \frametitle{Kommandoformat}
    \ti{Varje kommando skrivs på en rad:}

    \begin{GobbleCode}{6}
        kommandonamn -option1 -option2 ... argument1 ...
    \end{GobbleCode}

    \begin{itemize}
        \ii{Kommandot talar om vad som ska göras.}
        \ii{Optionerna modifierar kommandot på något sätt.}
        \ii{Argumenten är (oftast) filer eller kataloger som påverkas av kommandot.}
    \end{itemize}

    \halfblankline
    \ti{Exempel:}

    \halfblankline
    \begin{onlyenv}<.->
        \begin{tabular}{lp{60mm}}
            \code{scalac Calc.scala}        & Kompilerar \code{Calc.scala} med Scalakompilatorn                                                                                             \\
            \code{cp -i report.tex old.tex} & Kopierar \code{report.tex} till \code{old.tex}. \code{-i} betyder att systemet frågar om \code{old.tex} ska skrivas över, om den redan finns. \\
        \end{tabular}
    \end{onlyenv}
\end{frame}

\begin{frame}[fragile]
    \frametitle{Kommandon för filhantering}
    Exempel på kommandon för att hantera filer:

    \blankline
    \begin{tabular}{lp{8.5cm}}
        \code{cp orig kopia} & Kopiera filen \code{orig} till \code{kopia}.                                                                                                                                 \\
        \code{less fil}      & Skriv ut \code{fil} på skärmen, en sida i taget.                                                                                                                             \\
        \code{ls [-la] kat}  & Skriv ut en innehållsförteckning över katalogen \code{kat} (aktuell katalog om ingen katalog ges). \code{-l} skriv i ett längre format, \code{-a} skriv också ut punktfiler. \\
        \code{mv fil1 fil2}  & Döp om \code{fil1} till \code{fil2}. Om \code{fil2} är en katalog så flyttas filen till den katalogen.                                                                       \\
        \code{rm fil1 ...}   & Tag bort de angivna filerna.                                                                                                                                                 \\
    \end{tabular}
\end{frame}

\begin{frame}[fragile]
    \frametitle{Kataloghantering}
    Exempel på kommandon för att hantera kataloger:

    \blankline
    \begin{tabular}{lp{8.3cm}}
        \code{cd kat}    & Ändra aktuell katalog till \code{kat}. Utan argument blir det hemkatalogen (samma som \code{cd} \url{~}). \\
        \code{mkdir kat} & Skapa underkatalogen \code{kat} i den aktuella katalogen.                                                 \\
        \code{pwd}       & Skriv ut namnet på aktuell katalog.                                                                       \\
        \code{rmdir kat} & Tag bort \code{kat}. Man måste först ta bort alla filerna i katalogen.                                    \\
    \end{tabular}
\end{frame}

% \begin{frame}[fragile]
%     \frametitle{Filskydd 1}
%     Varje fil har en ägare. Ägaren identifieras med användarnamn och grupp, till exempel \code{ma1337no-s} i gruppen \code{students}.

%     \pindent Ägaren kan skydda filer så att andra inte kan komma åt dem, eller göra dem publika så att andra kan komma åt dem.

%     \pindent För att ta reda på filskydd och ägare (och storlek och datum) för filer kan man använda kommandot \code{ls} med optionen \code{l}, alltså \code{ls -l}:
%     \begin{GobbleCode}{6}
%         hacke-3{ma1337no-s}: ls -l
%         -rw-r-----  1 ma1337no-s  students  4940 aug 21 11:14 example.txt
%          (skydd)      (ägare)   (grupp)
%     \end{GobbleCode}

% \end{frame}


\begin{frame}[fragile,t]
    \frametitle{Filskydd 1/3}

    \ti{\textbf{Filägande:}}
    \begin{itemize}
        \is{Ägare identifieras via användarnamn och grupp\\
            \hspace{1em}(t.ex., \code{ma1337no-s} i \code{students})}
    \end{itemize}

    \ti{\textbf{Filsäkerhet:}}
    \begin{itemize}
        \is{Ändra åtkomsträttigheter för att skydda eller dela filer}
    \end{itemize}

    \ti{\textbf{Visa detaljer:}}
    \begin{onlyenv}<.->
        \begin{GobbleCode}{10}
            hacke-3{ma1337no-s}: ls -l
            -rw-r-----  1 ma1337no-s  students  4940 aug 21 11:14 example.txt
            (skydd)       (ägare)     (grupp)
        \end{GobbleCode}
    \end{onlyenv}

\end{frame}


\begin{frame}[fragile]
    \frametitle{Filskydd 2/3}
    \begin{GobbleCode}{6}
        -rw-r-----  1 ma1337no-s  students  4940 aug 21 11:14 example.txt
        u  g  o
    \end{GobbleCode}

    Tre kategorier av användare:

    \begin{description}
        \item[\code{u}] (user) filens ägare
        \item[\code{g}] (group) medlemmar i samma grupp som ägaren
        \item[\code{o}] (others) alla andra
    \end{description}

    Tre olika rättigheter:

    \begin{description}
        \item[\code{r}] (read) tillstånd att läsa
        \item[\code{w}] (write) tillstånd att skriva
        \item[\code{x}] (execute) tillstånd att exekvera program (för kataloger betyder \code{x} tillstånd att titta på innehållet i katalogen)
    \end{description}

\end{frame}

\begin{frame}[fragile,t]
    \frametitle{Filskydd 3/3}
    \ti{Kommando för att ändra filskydd:}

    \begin{onlyenv}<.->
        \begin{GobbleCode}{8}
            chmod skydd fil1 fil2 ...
        \end{GobbleCode}
    \end{onlyenv}

    \ti{Filskydd kan anges symboliskt, till exempel \code{u=rw,o=r}}

    \halfblankline
    \ti{Eller numeriskt, enligt:}

    \begin{onlyenv}<+->
        \begin{GobbleCode}{8}
            x = 001, w = 010, r = 100, rx = 101, rw = 110, rwx = 111
        \end{GobbleCode}
    \end{onlyenv}


    \ti{Siffrorna tolkar man sedan som binära tal, vilket ger:}

    \begin{onlyenv}<.->
        \begin{GobbleCode}{8}
            x = 1, w = 2, r = 4, rx = 5, rw = 6, rwx = 7
        \end{GobbleCode}
    \end{onlyenv}


    \ti{Fullständiga numeriska filskydd blir till exempel:}

    \begin{onlyenv}<.->
        \begin{GobbleCode}{8}
            604 <=> u=rw,o=r    755 <=> u=rwx,g=rx,o=rx
        \end{GobbleCode}
    \end{onlyenv}
\end{frame}

\begin{frame}[fragile]
    \frametitle{bash-finesser 1}
    Man kan editera den aktuella kommandoraden:

    \blankline
    \begin{tabular}{p{2.2cm}p{8cm}}
        \commandchar{$\leftarrow$\ $\rightarrow$} & Flytta markören på raden.                                                 \\
        \commandchar{Control-K}                   & Radera hela raden.                                                        \\
        \commandchar{Control-U}                   & Radera allt till vänster.                                                 \\
        \commandchar{Control-L}                   & Töm terminalfönstret.                                                     \\
        \commandchar{Tab}                         & Filnamnskomplettering (man behöver bara skriva början av ett långt namn). \\
    \end{tabular}

    \blankline
    \code{bash} håller reda på de senaste kommandona som utförts, och man kan få tillbaka gamla kommandon:

    \blankline
    \begin{tabular}{p{2.2cm}p{8cm}}
        \code{$\uparrow$} \code{$\downarrow$} & Bläddra bland tidigare kommandon.                  \\
        \code{!abc}                           & Gör om senaste kommando som inleds med \code{abc}. \\
        \code{!!}                             & Gör om senaste kommando.                           \\
    \end{tabular}
\end{frame}

\begin{frame}[fragile,t]
    \frametitle{bash-finesser 2}
    \ti{När man refererar till filer kan man utnyttja jokertecken (wildcards):}

    \blankline
    \begin{onlyenv}<.->
        \begin{tabular}{ll}
            \code{?} & motsvarar \emph{ett} godtyckligt tecken \\
            \code{*} & motsvarar 0--flera godtyckliga tecken   \\
        \end{tabular}
    \end{onlyenv}

    \blankline
    \ti{Antag att vi har följande filer i vår katalog:}
    \begin{onlyenv}<.->
        \begin{GobbleCode}{8}
            Test1.java  Test2.java  Test23.java  Final.java
            Test1.class Test2.class Final.class  FinalReport.tex
            Outline.tex
        \end{GobbleCode}
    \end{onlyenv}

    \ti{Då kan vi till exempel skriva:}
    \begin{onlyenv}<.->
        \begin{GobbleCode}{8}
            javac Test?.java (javac Test1.java Test2.java)
            rm *.class       (rm Test1.class Test2.class Final.class)
            gedit F*.tex     (gedit FinalReport.tex)
        \end{GobbleCode}
    \end{onlyenv}

\end{frame}

% \begin{frame}[fragile]
%     \frametitle{Initieringsfiler}
%     Många program läser en initieringsfil när de startas. Initieringsfilen innehåller inställningar för programmet. Namnen på initieringsfilerna inleds med punkt, så man ser normalt inte dessa filer när man gör \code{ls}.

%     \pindent \code{bash} läser initieringsfilen \code{.bash\_profile} när ett nytt terminalfönster skapas. Bra rader att ha i denna fil (gör att man alltid får frågor om man verkligen vill ta bort/skriva över filer):

%     \begin{Code}
%         alias rm='rm -i'
%         alias cp='cp -i'
%         alias mv='mv -i'
%     \end{Code}

%     Varning: kopiera inte andras initieringsfiler om du inte begriper vad som står i dem!
% \end{frame}

\begin{frame}[fragile,t]
    \frametitle{Initieringsfiler}

    \ti{\textbf{Om initieringsfiler:}}
    \begin{itemize}
        \is{Används ofta för att konfigurera program.}
        \ii{Filnamn börjar med punkt och är dolda, s.k. \emph{punktfiler}.}
    \end{itemize}

    \ti{\textbf{\code{bash} initieringsfil:}}
    \begin{itemize}
        \is{bash läser \code{.bash\_profile} (från hemkatalogen) när ett nytt terminalfönster öppnas.}
        \ii{Nyttiga alias för att förhindra oavsiktliga filoperationer:}
    \end{itemize}

    \begin{onlyenv}<.->
        \begin{Code}
            alias rm='rm -i'
            alias cp='cp -i'
            alias mv='mv -i'
        \end{Code}
    \end{onlyenv}

    \ti{\textbf{Varning:}}
    \begin{itemize}
        \is{Kopiera inte initieringsfiler utan att förstå innehållet!}
    \end{itemize}

\end{frame}


% \begin{frame}[fragile]
%     \frametitle{Vad är egentligen ett kommando?}
%     Kommandon är av två slag:

%     \begin{description}
%         \item[''inbyggda''] små kommandon som exekveras direkt i kommandotolken, till exempel \code{cd}, \code{pwd}, \code{exit}
%         \item[''vanliga program''] kommandotolken letar efter ett program med samma namn som kommandot
%     \end{description}

%     Kommandotolken letar efter program i de kataloger som anges i sökvägen (operativsystemvariabeln \code{PATH}, som man själv kan sätta med kommandot \code{export}). När man till exempel skriver \code{javac X.java} exekveras programmet \code{/usr/bin/javac}, eftersom katalogen \code{/usr/bin} finns i sökvägen.

%     \pindent När man ska exekvera ett eget program som finns i den aktuella katalogen skriver man:
%     \begin{Code}
%         ./programnamn
%     \end{Code}

%     \code{./} är nödvändigt eftersom \code{.} (den aktuella katalogen) inte finns i sökvägen.
% \end{frame}

\begin{frame}[fragile,t]
    \frametitle{Vad är egentligen ett kommando?}

    \ti{\textbf{Typer av kommandon:}}
    \begin{description}
        \di{Inbyggda}{Kör direkt i kommandotolken (t.ex. \code{cd}, \code{pwd}, \code{ls}).}
        \di{Vanliga program}{Kommandotolken söker efter programfiler i \code{PATH}.}
    \end{description}

    \ti{\textbf{Programexekvering:}}
    \begin{itemize}
        \is{\code{PATH} definierar var program letas upp (\code{export PATH=...} för anpassning).}
        \ii{Ex: \code{/usr/bin/javac} finns i \code{PATH}, så \code{javac X.java} går att köra.}
    \end{itemize}

    \ti{\textbf{Köra egna program:}}
    \begin{onlyenv}<.->
        \begin{Code}
            ./programnamn
        \end{Code}
    \end{onlyenv}
    \ti{Använd \code{./} för program i den aktuella katalogen, eftersom \code{.} inte ingår i \code{PATH}.}

\end{frame}


% \begin{frame}[fragile]
%     \frametitle{Processer}
%     Varje program som körs i Unix körs som en egen process som exekverar självständigt, ''parallellt'' med andra processer. Varje process körs några millisekunder, sedan får nästa process exekvera, osv. Detta hanteras av operativsystemet.

%     \pindent När man exekverar ett program från ett terminalfönster ''låser'' programmet fönstret tills det avslutas --- så vill man ofta att det ska fungera.

%     \pindent Men man kan också köra program ''i bakgrunden'' dvs frikopplade från terminalfönstret. Det gör man genom att skriva \code{\&} sist på kommandoraden, till exempel:

%     \begin{Code}
%         gedit example.txt &
%     \end{Code}

% \end{frame}
\begin{frame}[fragile,t]
    \frametitle{Processer}

    \ti{\textbf{Grundläggande om Processer:}}
    \begin{itemize}
        \ii{Varje program körs som en egen process, oberoende och ''parallellt'' med andra.}
        \ii{Processväxling sker kontinuerligt, hanteras automatiskt av operativsystemet.}
    \end{itemize}

    \ti{\textbf{Exekvering i Terminal:}}
    \begin{itemize}
        \ii{Standard: Kommandotolken väntar på programmet. Terminalen framstår som upptagen tills programmet avslutas.}
        \ii{Bakgrundskörning: Använd \code{\&} för att köra programmet frikopplat från terminalen.}
    \end{itemize}

    \ti{\textbf{Exempel:}}
    \begin{onlyenv}<.->
        \begin{Code}
            gedit example.txt &
        \end{Code}
    \end{onlyenv}
    \ti{\textit{Öppnar \code{gedit} i bakgrunden, terminalen blir omedelbart tillgänglig för nya kommandon.}}

\end{frame}


% \begin{frame}[fragile]
%     \frametitle{Fönstersystem, skrivbordsmiljö}
%     I Unix och Linux är fönstersystemet inte inbyggt (Unix utvecklades långt innan det fanns grafiska skärmar).

%     \pindent De flesta Unix- och Linuxsystem som kör på persondatorer använder fönstersystemet X Window System, som oftast kallas bara X. X gör det möjligt att ha fönster på skärmen, att flytta dem, ikonifiera dem, osv.

%     \pindent Exakt hur fönstren ska se ut och hur man arbetar med dem bestäms av \emph{fönsterhanteraren}, som också är ett separat program. Ovanpå fönsterhanteraren finns skrivbordsmiljön.

%     \pindent Det bästa (enda) sättet att lära sig att använda en skrivbordsmiljö är att träna!
% \end{frame}
\begin{frame}[fragile]
    \frametitle{Fönstersystem, skrivbordsmiljö}

    \ti{\textbf{Fönstersystem:}}
    \begin{itemize}
        \ii{I Unix och Linux är fönstersystemet inte inbyggt (Unix utvecklades långt innan det fanns grafiska skärmar).}
        \ii{Fönstersystemet X Window System (ofta kallat \emph{X}).}
    \end{itemize}

    \ti{\textbf{Fönsterhanterare och skrivbordsmiljö:}}
    \begin{itemize}
        \ii{Fönsterhanteraren bestämmer utseende och funktion hos fönstren.}
        \ii{Skrivbordsmiljön bygger vidare på fönsterhanteraren för en fullständig användarupplevelse.}
    \end{itemize}

    \ti{\textbf{Lära sig skrivbordsmiljön:}}
    \begin{itemize}
        \ii{För många av er är Linux nytt. Bästa sättet att bli bekant med en skrivbordsmiljö är att träna!}
    \end{itemize}

\end{frame}

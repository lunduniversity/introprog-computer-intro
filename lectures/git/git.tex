

\begin{frame}
    \frametitle{Vad är versionshantering?}
    
    motivation, t.ex.: ``Har du någonsin...'', mejlat filer fram och tillbaka, etc.
    
    Det är en form av versionshantering.
    
    Har du någonsin...
    förlorat en fil, skrivit över en fil, etc.
    
    lista fler exempel som är relevanta för studenter...
    
\end{frame}
    \frametitle{Versionshanteringssystem}
    
    \begin{frame}
    Hjälper oss att hantera komplexiteten det innebär att hålla reda på alla filer som hör till ett projekt -- källkod, dokumentation, etc.

    \blankline
    Låter oss gå tillbaka till äldre versioner av filerna om nödvändigt.

    \blankline
    Gör det möjligt för oss att arbeta parallellt med filerna i projektet utan att komma i vägen för varandra.
\end{frame}

\begin{frame}
    \frametitle{Versionshanteringssystem}
    
    \begin{block}{Vad är ett versionshanteringssystem?}
        Programvara som lagrar alla versioner som har existerat under projektets gång i någon form av databas.
    \end{block}
    
\end{frame}


\begin{frame}
    \frametitle{Versionshanteringssystem}

    % todo: add figures!
    
    \begin{block}{Todo}
        \begin{itemize}
        \item Centraliserade system (CVS, SVN)
        \begin{itemize}
            \item En central databas, ett repositorium (repo), håller reda på alla filer.
            \item Utvecklare \textit{checkar ut} en lokal kopia av filerna och arbetar med dem.
            \item När arbetet är klart checkar utvecklaren in filerna igen (kopierar dem till det centrala repot) varvid en ny version av projektet skapas.
        \end{itemize}
        \item Distribuerade system (Git)
        \begin{itemize}
            \item Finns inget centralt repo.
            \item I stället klonar man ett repo (gör en komplett kopia av repot).
            \item Ändringar kan hämtas (\textit{pull}), eller överföras till (\textit{push}), andra repon.
        \end{itemize}
        \end{itemize}
    \end{block}
    
\end{frame}



\title{Introduktion}
\section{Introduktion}


\begin{frame}[fragile=singleslide]
	\frametitle{Välkommen}
	\begin{itemize}
		\item Syftet med kursen? Introduktion inför andra kurser.
		\item Vem läser kursen?
		      \begin{itemize}
			      \item D-programmet.
			      \item C-programmet. Som del av EITA55
		      \end{itemize}
		\item Vad innehåller kursen?
		      \begin{itemize}
			      \item Tre ämnen, med vardera en föreläsning och en laboration
			      \item Unix och Linux
			      \item \LaTeX
			      \item Git och Github
		      \end{itemize}
		\item Läses tillsammans med EDAA45 (pgk, Programmering grundkurs)
	\end{itemize}
\end{frame}


\begin{frame}[fragile=singleslide]
	\frametitle{Schema och gruppindelning}

	\begin{itemize}
		\item Gruppindelning
		      \begin{itemize}
			      \item Gruppindelning samma som i EDAA45, se Canvas.
			      \item Ni som bara läser dod, ingen gruppindelning.
			      \item Kapten Alloc \url{https://fileadmin.cs.lth.se/pgk/kaptenalloc/}
		      \end{itemize}
		\item Schema
		\item Föreläsning alltid samma tid (tisdag 15:15), E:A
		\item Laborationer
		      \begin{itemize}
			      \item En labb per vecka, start imorgon!
			      \item Föreberedelseuppgifter, kontrollfrågor.
			      \item Laboration 1, Linux (Första laborationen redan imorgon!)
			      \item Laboration 2. \LaTeX
			      \item Laboration 3, Git
		      \end{itemize}
	\end{itemize}
\end{frame}



\begin{frame}[fragile=singleslide]
	\frametitle{Vid frågor}

	\begin{itemize}
		\item Ni kan alltid maila mig: \url{mattias.nordahl@cs.lth.se}
		\item Discord (för EDAA45). Hitta invite-länk på Canvas-sidan.
		\item Ni som inte går EDAA45, frivilligt att gå med.
	\end{itemize}
\end{frame}
% !TEX encoding = UTF-8 Unicode
\documentclass[a4paper]{article}

\usepackage[T1]{fontenc}     % För svenska bokstäver
\usepackage[utf8]{inputenc}  % Teckenkodning UTF8
\usepackage[swedish]{babel}  % För svensk avstavning och svenska
                             % rubriker (t ex "Innehållsförteckning")
\usepackage{fancyvrb}        % För programlistor med tabulatorer
\fvset{tabsize=4}            % Tabulatorpositioner
\fvset{fontsize=\small}      % Lagom storlek för programlistor

\usepackage{graphicx}

\title{Dokumentnamn}
\author{Nils Nilsson}
\date{1 augusti 1994}        % Blir dagens datum om det utelämnas

\begin{document}             % Början på dokumentet

\maketitle                   % Skriver ut rubriken som vi
                             % definierade ovan med \title, \author
                             % och eventuellt \date


Är vatten vått?


Här skriver jag något
\emph{viktigt}. Och
i Java har vi använt
klassen \texttt{Square}


DoD-kursen pågår under vecka
1--3 av läsperiod ht1. Tyvärr
är den inte längre \ldots $\cdots$


\quad Telefon: 046--222~80~38.
Dagens datum: \today.



\begin{itemize}
	\item första punkten
	\item här kommer den andra
	punkten i listan
	\end{itemize}

	\begin{enumerate}
		\item Nummer 1
		\begin{itemize}
			\item En sak
			\item en anna sak
		\end{itemize}
		\item nummer 2
		\item numm 3
	\end{enumerate}


\begin{description}
	\item[SimpleWindow] Ett ekel förnster
	\item[test] Test för någonting. Test för någonting. Test för någonting. Test för någonting. Test för någonting. Test för någonting. Test för någonting. Test för någonting. Test för någonting. Test för någonting. Test för någonting. Test för någonting. Test för någonting. Test för någonting. Test för någonting. Test för någonting. Test för någonting. Test för någonting.  
\end{description}

En rad

\vspace{20mm}

En anna rad


\begin{displaymath}
	x^2 = y^{r-2t^4}
\end{displaymath}

\begin{displaymath}
	sin^2 x = 4,
	\sin^2 x = 4
\end{displaymath}



\begin{displaymath}
	A=\left\{
	\begin{array}{cccc}
	a_{11} & a_{12} & \cdots & a_{1n}\\
	a_{21} & a_{22} & \cdots & a_{2n}\\
	\vdots & \vdots & \ddots & \vdots\\
	a_{n1} & a_{n2} & \cdots & a_{nn}\\
	\end{array}
	\right(
	\end{displaymath}

\end{document}               % Slut på dokumentet
